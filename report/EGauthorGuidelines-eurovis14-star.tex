% ---------------------------------------------------------------------------
% Author guideline and sample document for EG publication using LaTeX2e input
% D.Fellner, v1.13, Jul 31, 2008

\documentclass{egpubl-eurovis-star}
\usepackage{eurovis2014-star}

% --- for EuroVis
%\WsSubmission    % uncomment for submission to EuroVis
\WsPaper         % uncomment for final version of EuroVis contribution

\electronicVersion % can be used both for the printed and electronic version

% !! *please* don't change anything above
% !! unless you REALLY know what you are doing
% ------------------------------------------------------------------------

% for including postscript figures
% mind: package option 'draft' will replace PS figure by a filname within a frame
\ifpdf \usepackage[pdftex]{graphicx} \pdfcompresslevel=9
\else \usepackage[dvips]{graphicx} \fi

\PrintedOrElectronic

% prepare for electronic version of your document
\usepackage{t1enc,dfadobe}

\usepackage{egweblnk}
\usepackage{cite}

% For backwards compatibility to old LaTeX type font selection.
% Uncomment if your document adheres to LaTeX2e recommendations.
% \let\rm=\rmfamily    \let\sf=\sffamily    \let\tt=\ttfamily
% \let\it=\itshape     \let\sl=\slshape     \let\sc=\scshape
% \let\bf=\bfseries

% end of prologue

%\input{EGauthorGuidelines-body.inc} % commented by KK for ShareLaTeX use

% ---------------------------------------------------------------------
% EG author guidelines plus sample file for EG publication using LaTeX2e input
% D.Fellner, v1.17, Sep 23, 2010


\title[EG \LaTeX\ Author Guidelines]%
      {Bridge reconstruction in LiDAR point clouds}

% For anonymous conference submission, please enter your SUBMISSION ID.
\author[submission ID]{Timotej Kovač}

%% For the final version of your accepted paper, please enter the authors names and affiliations.
%\author[D. Fellner \& S. Behnke]
%       {D.\,W. Fellner\thanks{Chairman Eurographics Publications Board}$^{1,2}$
%        and S. Behnke$^{2}$
%        \\
%         $^1$TU Darmstadt \& Fraunhofer IGD, Germany\\
%         $^2$Institut f{\"u}r ComputerGraphik \& Wissensvisualisierung, TU Graz, Austria
%       }

% ------------------------------------------------------------------------

% if the Editors-in-Chief have given you the data, you may uncomment
% the following five lines and insert it here
%
% \volume{27}   % the volume in which the issue will be published;
% \issue{1}     % the issue number of the publication
% \pStartPage{1}      % set starting page


%-------------------------------------------------------------------------
\begin{document}

% \teaser{
%  \includegraphics[width=\linewidth]{eg_new}
%  \centering
%   \caption{New EG Logo}
% \label{fig:teaser}
% }

\maketitle

\begin{abstract}
TODO
\end{abstract}



\section{Introduction}

TODO




\section{Terrain reconstruction}

The majority of our effort was aimed towards reconstructing the surface underneath the bridge.
At the start we tried some basic approaches towards terrain reconstruction and got some insight into problems that we will have to solve.
These were:
\begin{itemize}
\item{how to obtain the area of the terrain under the bridge,}
\item{how to generate points on this area to best reconstruct the terrain and}
\item{how to deal with outliers and other vegetation, power lines and object that are adjecent to bridges.}
\end{itemize}

Our first attempt at solving this issue can be seen in figure~\ref{fig1}.

%TODO: Insert image of fail.

After some experimentation we have come to an algorhitm which proved very effective in reconstruction terrain under a bridge.
We have also chosen a very complex bridge example in order to make our approach as robust as possible.
Our example were two adjecent bridges that extended over a moving body of water at an angle.
Both also extended over a large portion of the terrain.
This example and its reconstruction attempt can be seen in figure~\ref{fig2}.

%TODO: Insert satelite picture of bridge and reconstruction.

Our approach:
\begin{itemize}
\item{1. define the bridge polygon from the SHP file and remove every point that is determined to belong to a bridge.}
\item{2. generate values x and y along the bridge in such a way that they run parallel to the valley bellow the bridge. }
\item{3. sample points from both sides of this line (terrain adjecent to the bridge). 
If there are no such points to be found sample the surrounging area of the terrain that has already been completed.}
\item{4. Process the sampled points to remove any objects that aren't terrain.}
\item{5. Interpolate the z coordinate with distance as a weight on sampled points from 4.}
\end{itemize}

This approach brings much better results than plain interpolation as it only takes into account the cloud points that are important for calculation.

\section{Bridge reconstruction}


\section{Issues}



\section{Conclusion}
TODO


%\bibliographystyle{eg-alpha}
\bibliographystyle{eg-alpha-doi}

\bibliography{egbibsample}

\end{document}

